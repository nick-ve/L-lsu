\Transcb{yellow}{blue}{Measurement of Cosmological distances in a Flat Universe}
\onecolumn
\begin{itemize}
\item Quite often it is necessary to know the physical distance of a certain object
\item[] E.g. to determine the actual energy output of a Gamma Ray Burst from observed fluence
\item WMAP \& Planck measurements : Consistent with a flat universe $\rightarrow k=0$
\item[] This yields for the Robertson-Walker metric describing a flat universe :
\item[] {\blue $\qquad \d s^{2}=(c\,\d t)^{2}-a^{2}(t)\left[
        (\d\sigma)^{2}+\sigma^{2}(\d\theta)^{2}+\sigma^{2}\sin^{2}(\theta)(\d\varphi)^{2}
        \right]$}
\item {\red Actual 3D spatial distances can only be measured correctly if $\d t=0$\\
      (i.e. at the same time)}
\item[] Due to the finite speed of light this is impossible for cosmological objects
\item {\blue Can we find an observable which can be related to the physical distance ?}
\item[] Use EM radiation for observations $\rightarrow \d s^{2} \equiv 0$
\item[] Distant objects may show emission/absorbtion lines $\rightarrow$ {\blue Use redshift $z$ as observable}
\item[$\ast$] {\red Can we relate the observed redshift $z$ to the physical distance of the object ?}
\item[] Consider the case that we observe at $t=t_{o}$ a light signal from a distant source that was emitted
        at some time $t<t_{o}$ 
\end{itemize}

\Tr
\begin{itemize}
\item Robertson-Walker for a flat universe~: $\displaystyle \d s^{2}=0
      \rightarrow {\blue \d\sigma=\frac{c}{a(t)}\,\d t} \qquad (1)$
\item For the cosmological redshift we have :
      $\displaystyle z(t)=\frac{a(t_{o})}{a(t)}-1
       \rightarrow \frac{\d z}{\d t}=\frac{-a(t_{0})\dot{a}(t)}{a^{2}(t)}$
\item[] Substitution of $\displaystyle \d t=\frac{-a^{2}(t)}{a(t_{o})\dot{a}(t)}\,\d z$ in (1) yields~:
        ${\blue \displaystyle \d\sigma=\frac{c}{a(t_{o})}\cdot\frac{a(t)}{\dot{a}(t)}\,\d z} \qquad (2)$
\item Friedmann-LeMa\^{i}tre for a flat universe : {\red $\displaystyle
      \left[\frac{\dot{a}(t)}{a(t)}\right]^{2}=\frac{8\pi G\rho(t)}{3}+\frac{\Lambda}{3}$}
\item[$\ast$] Total energy conservation within a comoving volume : $\rho(t)a^{3}(t)=\rho(t_{o})a^{3}(t_{o})$
\item[] Flat matter-dominated universe : $\rho(t_{o})=\rho_{c,0}\,\Omega_{M} \rightarrow$
       {\blue $\displaystyle \rho(t)=\left(\frac{a(t_{o})}{a(t)}\right)^{3}\frac{3H_{0}^{2}}{8\pi G}\Omega_{M}$}
\item[] Using {\blue $\Lambda=3H_{0}^{2}\,\Omega_{\Lambda}$} we obtain : ${\red \displaystyle
        \left[\frac{\dot{a}(t)}{a(t)}\right]^{2}
        =H_{0}^{2} \left[\Omega_{M}\left(\frac{a(t_{o})}{a(t)}\right)^{3}+\Omega_{\Lambda}\right]} \qquad (3)$
\end{itemize}

\Tr
\begin{itemize}
\item Using {\blue $\displaystyle \frac{a(t_{o})}{a(t)}=1+z$} in (3) yields for the Friedmann-LeMa\^{i}tre equation~:
\item[] \begin{center}
        {\red \shabox{$\displaystyle
        \left[\frac{\dot{a}(t)}{a(t)}\right]^{2}
        =H_{0}^{2} \left[\Omega_{M}\,(1+z)^{3}+\Omega_{\Lambda}\right] \qquad (4)$}}
        \end{center}
\item Combination of (2) and (4) yields~:
\item[] \begin{center}
        {\blue \shabox{$\displaystyle \d\sigma=
         \frac{c}{a(t_{o})H_{0}}\cdot\frac{1}{\sqrt{\Omega_{M}\,(1+z)^{3}+\Omega_{\Lambda}}}\,\d z \qquad (5)$}}
        \end{center}
\item Combination of (1) and (5) yields~:
        $\displaystyle c\,\d t=
         \frac{c}{H_{o}}\cdot\frac{a(t)}{a(t_{o})}\cdot\frac{1}{\sqrt{\Omega_{M}\,(1+z)^{3}+\Omega_{\Lambda}}}\,\d z$
\item[] \begin{center}
        {\blue \shabox{$\displaystyle c\,\d t=
         \frac{c}{H_{o}}\cdot\frac{1}{(1+z)}\cdot\frac{1}{\sqrt{\Omega_{M}\,(1+z)^{3}+\Omega_{\Lambda}}}\,\d z \qquad (6)$}}
        \end{center}
\item[] \colorbox{yellow}{The equations (5) and (6) provide the basis to define various cosmological distances}
\end{itemize}

\Tr
\onecolumn
\begin{center}
{\red Definitions of cosmological distances between source and observer}
\end{center}
% 
\begin{itemize}
\item[$\ast$] Light emitted at $(\sigma_{e},t_{e})$ is observed at $(0,t_{o})$ with a redshift $z_{o}$
\item[] Various cosmological distances could be of interest in studying the source
\begin{itemize}
\item What was the physical distance when the light was emitted ?
\item What is the physical distance when the light is observed ?
\item How long has the light traveled before it reached us ?
\item What is the luminosity of the source based on the observed flux ?
\end{itemize}
\item The {\blue Comoving Distance $D_{C}$}~: This is the (constant) distance in comoving coordinates
\item[] $\displaystyle D_{C}=\sigma_{e}=\int_{0}^{\sigma_{e}}\d\sigma$ and using (5) yields~:
\item[] \begin{center}
        {\red \shabox{$\displaystyle D_{C}(z_{o})=\frac{c}{a(t_{o})H_{0}}
         \int_{0}^{z_{o}}\frac{1}{\sqrt{\Omega_{M}\,(1+z)^{3}+\Omega_{\Lambda}}}\, \d z$}}
        \end{center}
\item[] Note : It is customary to "calibrate" the "$\sigma$-scale" by defining $a(t_{o}) \equiv 1$
\end{itemize}

\Tr
\onecolumn
\begin{itemize}
\item The {\blue Proper Distance $D_{P}$}~: (also called {\blue Physical Distance})
\item[] This is the (time-dependent) distance that would be measured by a ruler\\
        at a specific time $t \leq t_{o}$. In other words~: {\blue $D_{P}(t)=a(t)D_{C}$}
\item[] $\rightarrow \displaystyle {\blue D_{P}(t,z_{o})=\frac{a(t)}{a(t_{o})}\cdot\frac{c}{H_{0}}
         \int_{0}^{z_{o}}\frac{1}{\sqrt{\Omega_{M}\,(1+z)^{3}+\Omega_{\Lambda}}}\, \d z}$
\item[] So, $D_{P}(t)$ represents the physical distance of $D_{C}$ at a specific time $t$ in the past
\item[$\ast$] We get for the {\blue Proper Distance at the time of observation $D_{P}(t_{o},z_{o})$}~:
\item[] \begin{center}
        {\red \shabox{$\displaystyle D_{P}(t_{o},z_{o})=\frac{c}{H_{0}}
         \int_{0}^{z_{o}}\frac{1}{\sqrt{\Omega_{M}\,(1+z)^{3}+\Omega_{\Lambda}}}\, \d z$}}
        \end{center}
\item[] and for the {\blue Proper Distance at the time of emission $D_{P}(t_{e},z_{o})$}~:
\item[] \begin{center}
        {\red \shabox{$\displaystyle D_{P}(t_{e},z_{o})=\frac{1}{(1+z_{o})}\cdot\frac{c}{H_{0}}
         \int_{0}^{z_{o}}\frac{1}{\sqrt{\Omega_{M}\,(1+z)^{3}+\Omega_{\Lambda}}}\, \d z$}}
        \end{center}
\item[$\ast$] With $a(t_{o}) \equiv 1 \rightarrow
        D_{P}(t_{o},z_{o})=D_{C}(z_{o})$ and $D_{P}(t_{e},z_{o})=\frac{D_{C}(z_{o})}{(1+z_{o})}$
\end{itemize}

\Tr
\onecolumn
\begin{itemize}
\item The {\blue Light Travel Distance $D_{LT}$}~:
\item[] This is the distance that light has traveled to reach us from an object with redshift $z_{o}$
\item[] $\displaystyle D_{LT}=\int_{t_{e}}^{t_{o}} c\,dt$ which yields with (6)~:
\item[] \begin{center}
        {\red \shabox{$\displaystyle D_{LT}(z_{o})=\frac{c}{H_{0}}
         \int_{0}^{z_{o}}\frac{1}{(1+z)}\cdot\frac{1}{\sqrt{\Omega_{M}\,(1+z)^{3}+\Omega_{\Lambda}}}\, \d z$}}
        \end{center}
\item[$\ast$] The {\blue Size of the Universe $R_{U}$} can be expressed by {\blue $R_{U}=D_{LT}(z=\infty)$} 
\item The {\blue Light Travel Time $T_{L}$} is defined as~: {\blue $T_{L}(z_{o})=D_{LT}(z_{o})/c$}
\item[] So we have~:
\item[] \begin{center}
        {\red \shabox{$\displaystyle T_{L}(z_{o})=\frac{1}{H_{0}}
         \int_{0}^{z_{o}}\frac{1}{(1+z)}\cdot\frac{1}{\sqrt{\Omega_{M}\,(1+z)^{3}+\Omega_{\Lambda}}}\, \d z$}}
        \end{center}
\item[$\ast$] Note~: {\blue $T_{L}$} is also called the {\blue Look Back Time}
\item[$\ast$] The {\blue Age of the Universe $T_{U}$} can be expressed by {\blue $T_{U}=T_{L}(z=\infty)$} 
\end{itemize}

\Tr
\onecolumn
\begin{itemize}
\item The {\blue Luminosity Distance $D_{L}$}~:
\item[] This is the distance relating the intrinsic source luminosity ($L$) to the observed flux ($F$)
\item[$\ast$] Consider a source with an intrinsic luminosity of $L$ (erg sec$^{-1}$) of which we observe\\
              a flux $F$ (erg sec$^{-1}$ cm$^{-2}$) $\qquad \qquad$ Note~: 1 erg=$10^{-7}$ J
\item[$\ast$] Definition of luminosity distance~: {\blue $\displaystyle F=\frac{L}{4\pi D^{2}_{L}}$}
\item[] Concerning the observed flux $F$ we have the following cosmological effects
\begin{itemize}
\item Wavelength stretching $\rightarrow$ The observed energy is reduced by a factor $(1+z_{o})$
\item Time dilation $\rightarrow$ The observational time intervals are stretched by a factor $(1+z_{o})$
\item We should of course use the physical distance at the time of observation
\end{itemize}
\item[$\ast$] So we have {\blue $\displaystyle F=\frac{L}{4\pi D^{2}_{L}}=
                                  \frac{L}{4\pi (1+z_{o})^{2}D^{2}_{P}(t_{o},z_{o})}$} yielding~:
\item[] \begin{center}
        {\red \shabox{$\displaystyle D_{L}=(1+z_{o})D_{P}(t_{o},z_{o})=(1+z_{o})\cdot\frac{c}{H_{0}}
         \int_{0}^{z_{o}}\frac{1}{\sqrt{\Omega_{M}\,(1+z)^{3}+\Omega_{\Lambda}}}\, \d z$}}
        \end{center}
\end{itemize}

\Tr
\onecolumn
\begin{itemize}
\item[] \begin{center}{\red Some notes about the Luminosity Distance $D_{L}$}\end{center}
\item In case the observed {\blue Fluence $S$} (=time-integrated flux in erg/cm$^{2}$) is used in an analysis
\item[] one has to use $\sqrt{(1+z_{o})} \cdot D_P(t_{o},z_{o})$ as the corresponding distance
\item $D_{L}$ also relates the {\blue absolute magnitude $M$} of an astronomical object
\item[] to its observed {\blue apparent magnitude $m$} via~:
\item[] \begin{center}
        {\red \shabox{$\displaystyle M=m-5[^{10}\log(D_{L})-1] \quad \rightarrow \quad D_{L}=10^{[1+(m-M)/5]}$}}
        \end{center}
\item[] where $[D_{L}]$=parsec $({\rm pc})$ and
        $1\,{\rm pc} \approx 3.09 \cdot 10^{16}\,{\rm m} \approx$ 3.26 light year $({\rm ly})$
\end{itemize}