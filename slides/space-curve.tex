\Transcb{yellow}{blue}{Spatial curvature}
\onecolumn
\begin{itemize}
\item At large scales the universe is homogeneous and isotropic
\item[] If space is curved : curvature should be the same everywhere
        $\rightarrow$ {\blue constant curvature $K$}
\item[$\ast$] Metric can be determined in the same way as for the Schwarzschild case
\item[] Isotropic $\rightarrow$ use spherical coordinates and write {\red spatial metric} as
\item[] {\blue $\qquad \d s^{2}=f(r)(\d r)^{2}+r^{2}(\d\theta)^{2}+r^{2}\sin^{2}(\theta)(\d\varphi)^{2}$}
\item Reduce to 2-dim. case using $\theta=\frac{1}{2}\pi$ and $\d \theta=0$
      $\rightarrow {\blue \d s^{2}=f(r)(\d r)^{2}+r^{2}(\d\varphi)^{2}}$
\item[] For this 2D-surface~: $x_{1}=r \quad x_{2}=\varphi
        \rightarrow {\blue g_{\mu\nu}=\text{diag}(f(r),r^{2})}$
\item Gauss : $\displaystyle K=\frac{1}{2rf^{2}(r)} \cdot \frac{\d f(r)}{\d r}
               \rightarrow f(r)=\frac{1}{C-Kr^{2}} \quad$ ($C=$ constant)
\item[] Flat space~: $K \equiv 0$ and $f(r) \equiv 1 \rightarrow C=1$
\item[$\ast$] {\blue Spatial metric for constant curvature $K$}
\item[] \begin{center}
         {\red \shabox{$\displaystyle
         \d s^{2}=\frac{(\d r)^{2}}{1-Kr^{2}}+r^{2}(\d\theta)^{2}+r^{2}\sin^{2}(\theta)(\d\varphi)^{2}$}}
        \end{center}
\end{itemize}

\Tr
\begin{center}
{\red Some aspects of space with constant curvature $K$}
\end{center}
%
\begin{itemize}
\item Relation between {\blue radial coordinate $r$} and {\blue physical distance $D(r)$}
\item[] $\displaystyle D(r)=\int \d s=\int_{0}^{r} \frac{1}{\sqrt{1-Kr^{2}}}\,\d r
         =\frac{1}{\sqrt{K}} \arcsin(r\sqrt{K}) \quad (K>0)$
\item[] {\red \shabox{$\displaystyle D(r)=\frac{1}{\sqrt{K}}\arcsin(r\sqrt{K})$}} $\qquad$
        {\red \shabox{$\displaystyle r(D)=\frac{1}{\sqrt{K}}\sin(D\sqrt{K})$}}
\item Area $A$ of an $r$-sphere : $\displaystyle A=4\pi r^{2}=\frac{4\pi}{K}\sin^{2}(D\sqrt{K})$
\item[] Small $D \quad (D\sqrt{K} \ll 1) \rightarrow A \approx 4\pi D^{2} \quad$ (Euclidean value)
\item[] Large $D \quad (D\sqrt{K} \geqq 1) \rightarrow A$ increases more slowly than $4\pi D^{2}$
\item[$\ast$] {\blue In case $K>0$ the area $A$ of an $r$-sphere reaches two extreme values}
\item[] {\blue $\qquad D\sqrt{K}=(n+\frac{1}{2})\pi \rightarrow  A_{max}=\frac{4\pi}{K} \quad (n \in \mathbb{N})$}
\item[] {\blue $\qquad D\sqrt{K}=n\pi \rightarrow A_{min}=0$}
\item[] {\red Positively-curved $(K>0)$ space is closed}
\end{itemize}

\Tr
\begin{itemize}
\item For $K<0$ we get
\item[] $\displaystyle D(r)=\int \d s=\int_{0}^{r} \frac{1}{\sqrt{1+|K|r^{2}}}\,\d r
         =\frac{1}{\sqrt{|K|}} {\rm \,arcsinh}(r\sqrt{|K|})$
\item[] {\red \shabox{$\displaystyle D(r)=\frac{1}{\sqrt{|K|}}{\rm \,arcsinh}(r\sqrt{|K|})$}} $\qquad$
        {\red \shabox{$\displaystyle r(D)=\frac{1}{\sqrt{|K|}}\sinh(D\sqrt{|K|})$}}
\item So the area of an $r$-sphere becomes~: $\displaystyle A=\frac{4\pi}{|K|}\sinh^{2}(D\sqrt{|K|})$
\item[] $A$ increases faster than in flat space
\item[] $A$ increases to $\infty$ when $D \rightarrow \infty$
\item[$\ast$] {\red Negatively-curved $(K<0)$ space is open}
\item {\blue Can we now also derive the deformation of the time coordinate ?}
\item[] Depends on the space-time evolution of the universe
\item[] $r \rightarrow r(t)$ and consequently $K \rightarrow K(t)$
\item[$\ast$] Need for observational data w.r.t. the space-time evolution of the universe
\end{itemize}
