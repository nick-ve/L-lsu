\Transcb{yellow}{blue}{The Friedmann-LeMa\^{i}tre equation}
\onecolumn
\begin{itemize}
\item Determination of $a(t)$ and $k$ : need energy distr. and full treatment of Einstein's equations
\item[] Energy distribution not very well known $\rightarrow$ Need some assumptions
\item[] Assume various expressions for $a(t) \rightarrow$ Various models for the universe
\item[] Make predictions for certain observables within these models
\item[] Measure the observables in real life $\rightarrow$ Accept or reject corresponding model(s)
\item[$\ast$] {\red We will use classical arguments to derive an expression for $a(t)$}
\item Consider a test particle with mass $m$ at a location $\vec{r}$ w.r.t. some origin $O$
\item[] $m$ feels gravitational potential of the mass $M$ contained in the $r$-sphere as located in $O$
\item[] Spherical symmetry : $\vec{g}(\vec{r})=\vec{g}(r)$ and $\vec{\nabla}\times\vec{g}(r)=0$
\item[] {\red Conservative gravitational potential $\rightarrow E_{tot}$ of $m$ is constant}
\item[] $\rightarrow {\blue \frac{1}{2}mv^{2}-{\displaystyle \frac{GMm}{r}}=C} \qquad (C=\text{ constant})$
\item Uniform density : $M=\frac{4}{3}\pi r^{3}\rho
                         \rightarrow {\blue \frac{1}{2}mv^{2}-\frac{4}{3}\pi Gm\rho r^{2}=C}$
\end{itemize}

\Tr
\begin{itemize}
\item Using {\blue $\vec{v}(t)=\dot{\vec{r}}(t)$} and {\blue $\vec{r}(t)=a(t)\sigma$}
      we can write the previous formula as
\item[] $\qquad \frac{1}{2}m\dot{a}^{2}(t)-\frac{4}{3}\pi Gm\rho(t)a^{2}(t)=C/\sigma^{2}$
\item Defining $k \equiv 2C/(\sigma^{2}m)$ yields :  $\dot{a}^{2}(t)-\frac{8}{3}\pi G\rho(t)a^{2}(t)=k$
\item[] which is called the {\blue Friedmann-LeMa\^{i}tre equation}
\item[] \begin{center}
        {\red \shabox{
        $\displaystyle \left[\frac{\dot{a}(t)}{a(t)}\right]^{2}=\frac{8\pi G\rho(t)}{3}-\frac{k}{a^{2}(t)}$}}
        \end{center}
\item[$\ast$] This is the exact form obtained from Einstein's equations for a homogeneous and isotropic
              universe with {\blue total energy density $\rho(t)$} and {\blue curvature $k$}
\item The Friedmann-LeMa\^{i}tre equation describes the dynamical evolution of the universe
\item[] {\red Evolution of the universe completely determined by $\rho(t)$ and $k$}
\item Note : $\displaystyle H(t)=\frac{\dot{a}(t)}{a(t)} \rightarrow
              {\blue H^{2}(t)=\frac{8\pi G\rho(t)}{3}-\frac{k}{a^{2}(t)}}$
\end{itemize}