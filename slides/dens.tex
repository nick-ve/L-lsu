\Transcb{yellow}{blue}{Critical density, Dark energy and Cosmological constant}
\onecolumn
\begin{itemize}
\item Universe can be closed $(k>0)$, flat $(k=0)$ or open $(k<0)$
\item[] $\rightarrow$ {\blue Flat universe can be considered as a special (critical) case}
\item Flat universe : $\displaystyle H^{2}(t)=\frac{8\pi G\rho(t)}{3} \rightarrow$
      {\blue critical density : $\displaystyle \rho_{c}(t) \equiv \frac{3H^{2}(t)}{8\pi G}$}
\item[$\ast$] Indicate present day values as $\rho_{0}$ and $H_{0} \rightarrow
      \displaystyle {\red \rho_{c,0}=\frac{3H_{0}^{2}}{8\pi G}}$
\item {\blue What are the components which make up the total energy density $\rho(t)$ ?}
\item[] Universe contains {\red matter} $\rightarrow$ {\red matter energy density $\rho_{m}(t)$}
\item[] Universe contains {\red radiation} $\rightarrow$ {\red radiation energy density $\rho_{r}(t)$}
\item[] Unknown {\red influence of the vacuum} $\rightarrow$ {\red vacuum energy density $\rho_{v}(t)$}
\item {\blue Fractional density parameters : $\displaystyle
       \Omega_{m}(t) \equiv \frac{\rho_{m}(t)}{\rho_{c}(t)} \quad
       \Omega_{r}(t) \equiv \frac{\rho_{r}(t)}{\rho_{c}(t)} \quad
       \Omega_{v}(t) \equiv \frac{\rho_{v}(t)}{\rho_{c}(t)}$}
\item[] $\displaystyle \Rightarrow {\red \rho(t)=\rho_{m}(t)+\rho_{r}(t)+\rho_{v}(t) \qquad
         \Omega(t)=\frac{\rho(t)}{\rho_{c}(t)}=\Omega_{m}(t)+\Omega_{r}(t)+\Omega_{v}(t)}$
\item Note : {\blue Flat universe $\leftrightarrow \Omega(t)=1$}
\end{itemize}

\Tr
\begin{itemize}
\item The vacuum energy density is also called {\blue dark energy density}
\item[] Note : Don't confuse this dark energy density with dark matter !
\item[$\ast$] Einstein's equations allow only a {\red time-independent vacuum effect} :
        $\rho_{v}(t) \rightarrow \rho_{v}$
\item[] $\displaystyle \rightarrow {\blue \left[\frac{\dot{a}(t)}{a(t)}\right]^{2}=
         \frac{8\pi G\rho_{m}(t)}{3}+\frac{8\pi G\rho_{r}(t)}{3}+\frac{8\pi G\rho_{v}}{3}
         -\frac{k}{a^{2}(t)}}$
\item It is custom to define : ${\blue \rho(t) \equiv \rho_{m}(t)+\rho_{r}(t)} \text{~~~and~~~} \blue{\Lambda \equiv 8\pi G\rho_{v}}$
\item[] so that the Friedmann-LeMa\^{i}tre equation can be written as
\item[] \begin{center}
        {\red \shabox{$\displaystyle \left[\frac{\dot{a}(t)}{a(t)}\right]^{2}=
        \frac{8\pi G\rho(t)}{3}+\frac{\Lambda}{3}-\frac{k}{a^{2}(t)}$}}
        \end{center}
\item[] where {\blue $\Lambda$} is called the {\blue Cosmological constant}
\item Currently $\rho_{r,0} \ll \rho_{m,0} \rightarrow$ Only $2$ relevant parameters :
      {\blue $\displaystyle \Omega_{M} \equiv \frac{\rho_{m,0}}{\rho_{c,0}} \qquad
      \Omega_{\Lambda} \equiv \frac{\rho_{v}}{\rho_{c,0}}$}
\item[$\ast$] Planck results (2018) : {\red $\Omega_{M}=0.315 \pm 0.007 \qquad \Omega_{\Lambda}=0.685 \pm 0.007$}
\item[] {\blue It seems that we live in a flat universe !}
\end{itemize}
